
        \documentclass{article}
        \usepackage{todonotes}
        \usepackage{array}
        \usepackage{longtable}
        \usepackage{enumitem}
        \usepackage{tikz}
        \pagestyle{empty}
        \thispagestyle{empty}
        \usetikzlibrary{shadows}
        \usetikzlibrary{decorations}
        \usetikzlibrary{shapes}
        \usetikzlibrary{arrows}
        \usetikzlibrary{calc}
        \usetikzlibrary{fit}
        \usetikzlibrary{backgrounds}
        \usetikzlibrary{positioning}
        \usetikzlibrary{chains}
        \usetikzlibrary{scopes}
        \renewcommand*\familydefault{\sfdefault} %% Only if the base font of the document is to be sans serif
        \usepackage[normalem]{ulem}
        \newenvironment{events}
        {\begin{tabular}{>{\centering}m{\tableWidth}}}
        {\end{tabular}}
        \newenvironment{states}
        {\begin{minipage}{\tableWidth}\raggedright\begin{description}[align=left,leftmargin=1em,noitemsep,labelsep=\parindent]}
        {\end{description}\end{minipage}}
        \begin{document}
        
% Event macro definitions
% ------------------------------------------------------------------------
\newcommand{\Ezero}{\begin{events}
go(\allowbreak{}hero, evilLair): herosJourney\\
intHerosJourney(\allowbreak{}hero, villain, evilLair, home): herosJourney\\
\end{events}}

\newcommand{\Eone}{\begin{events}
intHerosJourney(\allowbreak{}hero, villain, evilLair, home): herosJourney\\
kill(\allowbreak{}hero, villain): herosJourney\\
\end{events}}

\newcommand{\Etwo}{\begin{events}
go(\allowbreak{}hero, home): herosJourney\\
intHerosJourney(\allowbreak{}hero, villain, evilLair, home): herosJourney\\
\end{events}}

% State macro definitions
% ------------------------------------------------------------------------
\newcommand{\Szero}{\begin{states}
\item\textbf{\sout{phase(\allowbreak{}herosJourney, inactive): herosJourney}}
\item\textbf{{place(\allowbreak{}home, home): herosJourney}}
\item\textbf{{place(\allowbreak{}evilLair, evilLair): herosJourney}}
\item\textbf{{role(\allowbreak{}villain, villain): herosJourney}}
\item\textbf{{role(\allowbreak{}hero, hero): herosJourney}}
\item\end{states}}

\newcommand{\Sone}{\begin{states}
\item\textbf{\sout{phase(\allowbreak{}herosJourney, phaseA): herosJourney}}
\item\textbf{{place(\allowbreak{}home, home): herosJourney}}
\item\textbf{{place(\allowbreak{}evilLair, evilLair): herosJourney}}
\item\textbf{{role(\allowbreak{}villain, villain): herosJourney}}
\item\textbf{{role(\allowbreak{}hero, hero): herosJourney}}
\end{states}}

\newcommand{\Stwo}{\begin{states}
\item\textbf{\sout{phase(\allowbreak{}herosJourney, phaseB): herosJourney}}
\item{{place(\allowbreak{}evilLair, evilLair): herosJourney}}
\item{{place(\allowbreak{}home, home): herosJourney}}
\item{{role(\allowbreak{}hero, hero): herosJourney}}
\item{{role(\allowbreak{}villain, villain): herosJourney}}
\end{states}}

\newcommand{\Sthree}{\begin{states}
\item{{place(\allowbreak{}home, home): herosJourney}}
\item{{place(\allowbreak{}evilLair, evilLair): herosJourney}}
\item{{role(\allowbreak{}villain, villain): herosJourney}}
\item{{role(\allowbreak{}hero, hero): herosJourney}}
\end{states}}

% Institutional trace
% ------------------------------------------------------------------------
\newlength{\tableWidth}
\setlength{\tableWidth}{5cm}

\begin{longtable}{@{}l@{}}
\resizebox{\textwidth}{!}{
\begin{tikzpicture}
[
start chain=trace going right,
start chain=state0 going down,
start chain=state1 going down,
start chain=state2 going down,
start chain=state3 going down,
node distance=1cm and 5.2cm
]
{{ [continue chain=trace]
\node[circle,draw,on chain=trace](i0) {$S_{0}$};
\node[circle,draw,on chain=trace](i1) {$S_{1}$};
\node[circle,draw,on chain=trace](i2) {$S_{2}$};
\node[circle,draw,on chain=trace](i3) {$S_{3}$};
{ [continue chain=state0 going below]
\node [on chain=state0,below=of i0,rectangle,draw,inner frame sep=0pt] (s0) {\Szero};} % end node and chain
\draw (i0) -- (s0);
{ [continue chain=state1 going below]
\node [on chain=state1,below=of i1,rectangle,draw,inner frame sep=0pt] (s1) {\Sone};} % end node and chain
\draw (i1) -- (s1);
{ [continue chain=state2 going below]
\node [on chain=state2,below=of i2,rectangle,draw,inner frame sep=0pt] (s2) {\Stwo};} % end node and chain
\draw (i2) -- (s2);
{ [continue chain=state3 going below]
\node [on chain=state3,below=of i3,rectangle,draw,inner frame sep=0pt] (s3) {\Sthree};} % end node and chain
\draw (i3) -- (s3);
}}
\draw[-latex,thin](i0) -- node[above]{\Ezero}(i1);
\draw[-latex,thin](i1) -- node[above]{\Eone}(i2);
\draw[-latex,thin](i2) -- node[above]{\Etwo}(i3);
\end{tikzpicture}}
\end{longtable}
\end{document}
