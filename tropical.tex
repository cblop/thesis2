% HACK: predicted poms remaining: 73 / 79

\chapter{TropICAL: A DSL for Story Tropes}
\label{cha:tropical}
% DONE introduction: 1
% DONE motivation: 1

This chapter describes the design and implementation of TropICAL (the TROPe
Interactive Chronical Action Language), our Domain Specific Language for
describing story tropes. The purpose of this language is to allow for the
creation of interactive narratives through the description of tropes in a
constrained natural language. The tropes created through the language are
designed to be reusable components that can go into a ``library'', from which
story authors can choose the tropes best suited to the particular story that they
are creating.

The motivation for the creation of TropICAL is the lack of any methods of
creating interactive narrative that are suitable for non-programmers to use. The
systems described in the literature review in
chapter~\ref{cha:literature-review}, such as any using the Mimesis
architecture~\cite{young2004architecture}, a drama manager and planner such as
in Fa\c{c}ade's system~\cite{mateas2003faccade}, or linear logic as in Ceptre's
system~\ref{martens2015ceptre}, all require the story author to be familiar with
planner-based systems or the description of formal logics. The purpose of
TropICAL is to greatly reduce the barrier to the creation of interactive stories
by allowing authors to describe the components of their story using constrained
natural language. In fact, many authors using our system may not even need to
write their own tropes using TropICAL, and will instead simply select the tropes
that they need for their story from a pre-created ``library'' of tropes. This
process is facilitated through the ``StoryBuilder'' user interface described in
chapter~\ref{cha:storybuilder}

This section describes TropICAL's constrained natural
language syntax in section~\ref{sec:t-constrained}, the features of the language in section~\ref{sec:t-features}, 
the formal grammar used to
describe its syntax in section~\ref{sec:t-grammar}, how ``nested
tropes'' are created in section~\ref{sec:t-nested}, the intermediate data structure used between parsing and
code generation steps (section~\ref{sec:t-intermediate}), its compilation
to InstAL code (section~\ref{sec:t-codegen}), some samples of generated InstAL
institutions (section ~\ref{sec:t-tropes}), answer set
generation (section~\ref{sec:t-asp}), and finally its extension for the description of legal policies
(section~\ref{sec:t-legal}).


\section{Controlled Natural Language Syntax}
\label{sec:t-constrained}
% DONE intro: 1
% DONE describe ACE: 2
% DONE describe Inform 7: 2

TropICAL uses a \emph{Controlled Natural Language} (also referred to as
\emph{Constrained Natural Language}) syntax, which means that it superficially
resembles natural language English, but is only a subset.

There are two types of Controlled Natural Language (CNL). The first type are \emph{naturalistic} languages
such as ASD Simplified Technical English~\cite{asd2007simplified}, designed to
be used in documentation for technical industries such as aerospace or defense,
or Ogden Basic English~\cite{ogden1944basic}, a simplified language for teaching
English as a second language. This type of Controlled Natural Language merely
describe a subset of the parent language. The other type of CNL are
\emph{formalistic}, with a formal
syntax and semantics, which can be mapped to rules in other formal languages
such as first-order logic. Attempto Controlled English~\cite{fuchs1996attempto}
is an example of this formal type of CNL, and the one which forms the basis of
our syntax for TropICAL.

\subsection{Attempto Controlled English}

Attempto Controlled English (ACE) is a controlled natural language that is also
a formal language, and was created by the University of Zurich in 1995. It is
still under development, with version 6.7 of the language announced in 2013. ACE
has been used in a wide variety of fields, such as software specifications,
ontologies, medical documentation and planning.

ACE's vocabulary has three components:

\begin{compactitem}
  \item predefined function words
  \item predefined fixed phrases (\emph{there is}, \emph{it is false that}, ...)
  \item content words (nouns, proper nouns, verbs, adjectives, adverbs)
\end{compactitem}

The Attempto Parsing Engine (APE) has a lexicon of content words, and users can
define their own content words. User-defined content words take precedence over
the built-in lexicon.

ACE's syntax is expressed as a set of construction rules (admissible ACE
sentence structures), with syntactically
correct sentences described as a set of interpretation rules, which contain the
actual semantics of the sentences.

The simplest ACE sentences follow a \emph{subject + verb + complements +
  adjuncts} structure:

\begin{lstlisting}
  A dragon sleeps.
  The sky is blue.
  A princess owns a castle.
  A king gives a sword to a knight.
\end{lstlisting}

In this structure, every sentence must have a subject and a verb. Sentences that
do not contain a verb are expressed with the \emph{there is} structure:

\begin{lstlisting}
  There is a kettle.
  There are 3 balls.
\end{lstlisting}

Details can be added to these sentences through the use of adjectives, and the
sentences can be joined together with the \emph{and} and \emph{then} keywords.

Other ways of modifying the sentences include \emph{negation}, adding
\emph{quantifiers}, or making the sentences \emph{interrogative} or
\emph{imperative}. We do not go into further details in this thesis, as we are
using ACE simply as an inspiration for our syntax, rather than following its
design entirely. For more details, see the ``ACE in a Nutshell'' page of its
documentation for an overview of the language~\cite{ace-nutshell}.

\subsection{Inform 7}

Inform is a programming language for the creation of Interactive Fiction (also
called \emph{text adventure games}). All versions of Inform generate
\emph{Z-code}, which is interpreted by the \emph{Z-machine} virtual machine for
interactive fiction.

The syntax of the language has changed multiple times since its creation in
1993. The version of the language that we are most interested in is Inform 7~\cite{reed2010creating},
which is its most recent incarnation. Prior to Inform 7, the language used
traditional programming models such as procedural and object-oriented paradigms.
With version 7, however, its creator Graham Nelson redesigned its
syntax completely to allow authors to create their stories using controlled
natural language, so that the experience of story creation became closer to that
of writing a book.

The simplest possible game authored with Inform 7 would be:

\begin{lstlisting}
"Hello World" by "I.F. Author"

The world is a room.

When play begins, say "Hello, world."
\end{lstlisting}

An extended example, Will Crowther's cave exploration simulation, would be
described like this:

\begin{lstlisting}
"Cave Entrance"

The Cobble Crawl is a room. "You are crawling over cobbles in a low passage. There is a dim light at the east end of the passage."

A wicker cage is here. "There is a small wicker cage discarded nearby."

The Debris Room is west of the Crawl. "You are in a debris room filled with stuff washed in from the surface. A low wide passage with cobbles becomes plugged with mud and debris here, but an awkward canyon leads upward and west. A note on the wall says, 'Magic word XYZZY'."

The black rod is here. "A three foot black rod with a rusty star on one end lies nearby."

Above the Debris Room is the Sloping E/W Canyon. West of the Canyon is the Orange River Chamber.
\end{lstlisting}

Although Inform 7 is not based directly on ACE, we can see many of ACE's
features in it. For example, it uses the same syntax for simple statements such
as \emph{X is a Y}. Most statements in the listing above are pairs of sentences,
with the first in each pair declaring the existence of an object or room, and
the second being an uninterpreted string that is printed out as the description
of the defined object.

Before version 7, Inform was already widely used as a language for the creation
of Interactive Fiction. Given its continued popularity after the switch to
a controlled natural language syntax, it appears that the new programming
paradigm has been successful.

The wide adoption of Inform 7 by story authors with no other programming
experience motivates our decision to create a controlled natural language syntax
for TropICAL, our own trope-based programming language. The intended audience is
the same, so an additional benefit would be that story authors who can code with
Inform 7 would also be able to use TropICAL.

\section{Features}
\label{sec:t-features}
% TODO intro: 1
% TODO list features: 1
% TODO tech used: 1
% TODO relate features to the requirements: 2

The way in which our TropICAL language is used is as follows:

\begin{compactitem}
  \item It is based around the idea of using \emph{story tropes} as reusable
    components that can be combined to create stories
  \item It uses \emph{controlled natural language} to describe these tropes
  \item The controlled natural language is then parsed and compiled to a data
    structure that describes the formal features of each trope
  \item The intermediate data structure is then used to generate InstAL code
  \item The InstAL code is compiled to AnsProlog, an Answer Set Programming
    language
  \item The AnsProlog code is then run through
    \emph{clingo}~\cite{gebser2011potassco}, an answer set solver, to generate
    all the possible sequences of events that can occur in the story.
\end{compactitem}


\section{EBNF Grammar}
\label{sec:t-grammar}
% TODO intro: 1
% TODO instaparse description: 1
% TODO simplified grammar from code: 2
% TODO format grammar: 2

\begin{lstlisting}
trope = (tropedef (<whitespace> (situationdef / alias / sequence))+ <'\\n'?>) | ((situationdef / alias / sequence)+ <'\\n'?>)

<tropedef> = label <' is a ' ('trope' / 'policy') ' where:\\n'>

alias =
    (character <' is also '> character <'\\n'?>) | (object <' is '> object <'\\n'?>)

situation =
    <'When '> event <':'>

fluent =
    object <' is '> adjective

adjective = word

object = name

outcome =
    (<'\\n' whitespace whitespace> (event | obligation | happens) (or? | if?) <'\\n'?>)+

happens =
    <the?> subtrope <(' happens' / ' policy applies') '.'?>


block =
    <the?> subtrope <' policy does not apply' / ' does not happen'> <'.'?>


sequence =
    ((event | norms | happens | block)  <'\\n'?> (or* | if*)) | ((event | norms | happens | block) (<'\\n' whitespace+ 'Then '> (block / norms / event / obligation / happens) (or* | if*))*)*

situationdef = situation (<'\\n'> <whitespace> norms | <'\\n'> <whitespace whitespace> consequence)+ <'\\n'?>

or =
    <'\\n' whitespace+ 'Or '> (event | norms)

if =
    <'\\n' whitespace+ 'If '> (event | norms)

event =
    (<'The '>? character <' is'>? <' '> (move / task)) / give / sell / tverb / meet / kill / pay


give =
    character <' gives ' ('the ' / 'a ' / 'an ')?> item <' to '>? <'a ' / 'an '>? character

sell =
    character <' sells ' ('the ' / 'a ' / 'an ')?> item <' to '>? <'a ' / 'an '>? character

tverb =
    character <' '> verb <(' the ' / ' a ' / ' an ')?> item <' to '>? <'a ' / 'an '>? character

bverb =
    verb <(' the ' / ' a ' / ' an ')?> item <' to '>? <'a ' / 'an '>? character

cverb =
    words <' the ' / ' a ' / ' an '> (object / character)

meet =
    character <' meets '> character
kill =
    character <' kills '> character
pay =
    <'pay '> character

norms = permission | rempermission | obligation

violation = norms

character = name

subtrope = <'\"'> words <'\"'>

label = <'\"'> words <'\"'>

move = mverb <' '> <'to '?> place
mverb = 'go' / 'goes' / 'leave' / 'leaves' / ('return' <!' the'>) / 'returns' <!' the'> / 'at' / 'come' / 'comes'
verb = word
place = name

<pverb> = verb (<' '> verb)*

rempermission = character <' may not '> (move / pay / bverb / cverb / task) <'\\n'?>
permission = character <' may '> (move / pay / bverb / cverb / task) <'\\n'?>
obligation = character <' must '> (move / pay / bverb / cverb / pverb / task) (<' before '> deadline)? (<'\\n' whitespace+ 'Otherwise, '> <'the '?> violation)? <'.'?> <'\\n'?>

deadline = consequence

task = pverb <' '> role-b / verb / (verb <(' the ' / ' a ' / ' an ')> item) / (verb <' '> item)
role-b = name

pverb = 'kill' / 'kill'<'s'> / 'refund'<'s'> / 'refund'

consequence =
    [<'The ' / 'the '>] character <' will '>? <' '> (move / bverb / cverb / item)
    | [<'The ' / 'the '>] item <' '> verb

item = [<'The ' / 'the '>] word

<whitespace> = #'\\s\\s'

<name> = [<'The ' / 'the '>] cwords
<words> = word (<' '> word)*
<cwords> = cword (<' '> ['of'<' '>] cword)*
<cword> = #'[A-Z][0-9a-zA-Z\\-\\_\\']*'
<the> = <'The ' | 'the '>
<word> = #'[0-9a-zA-Z\\-\\_\\']*'
\end{lstlisting}

\section{Nested Tropes via Bridge Institution}
\label{sec:t-nested}
% TODO why nested tropes: 1
% TODO explanation of bridge institutions: 2

\section{Intermediate Data Structure}
\label{sec:t-intermediate}
% TODO why do we have an intermediate data structure? (possible other uses,
% different parsers): 1
% TODO describe the hash map: 2

\section{InstAL Code Generation}
\label{sec:t-codegen}
% TODO intro: 1
% TODO tech discussion: 1

\subsection{Initiation}
% TODO before / after snippets: 2
% TODO explanation: 2

\subsection{Termination}
% TODO before / after snippets: 2
% TODO explanation: 2

\subsection{Generation}
% TODO before / after snippets: 2
% TODO explanation: 2

\subsection{Initial Conditions}
% TODO before / after snippets: 2
% TODO explanation: 2

\subsection{Bridge Institution}
% TODO before / after snippets: 2
% TODO explanation: 2

\section{Generated Trope Samples}
\label{sec:t-tropes}
% TODO intro for heros journey trope: 1
% TODO run through generated code: 3

% TODO intro for evil empire trope: 1
% TODO run through generated code: 3

\section{Adding Constraints}
\label{sec:t-constraints}
% TODO ASP intro: 1
% TODO constraints explanation: 2
% TODO output: 2

\section{Answer Set Generation}
\label{sec:t-asp}
% TODO instalquery intro & explanation: 2
% TODO generate for hero's journey: 4
% TODO generate for evil empire: 4
% TODO generate for both: 4

\section{Extending for Legal Policies}
\label{sec:t-legal}
% TODO intro: 1
% TODO explanation from JURIX: 2

\subsection{Example Policies}
% TODO examples from JURIX: 2

% TODO outro: 1